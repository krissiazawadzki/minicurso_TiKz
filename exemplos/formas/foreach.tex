\documentclass{standalone}
\usepackage{tikz}
\usetikzlibrary{shapes, calc}



\begin{document}


\begin{tikzpicture}[
vertex_style/.style={circle,shading=ball,ball color=red,draw=red!80!white,drop shadow={opacity=0.4}},
node_spin/.style={circle, shading=ball, ball color=gray!10!white, draw=none, minimum width=1.5cm,inner sep=0.1cm, opacity=0.7},
edge_style/.style={ultra thick, black,drop shadow={opacity=0.4}},
		hora/.style={rectangle, minimum width=1.75cm, minimum height=1.75cm, rounded corners=2mm, top color=gray!10, bottom color=brown!50}, font=\huge \bfseries, 
		]
 
\begin{scope}[xshift=0cm, rotate=0]

	\def\lados{12};
	\node[regular polygon, regular polygon sides=\lados, minimum size=15.cm, draw=brown, dashed, line width=2pt, rotate=15, transform shape, opacity=0.3] at (0,0) (A) {};
	


\foreach[count=\acc from 0] \i in {1,...,\lados}{
	
		\node[regular polygon, regular polygon sides=5, minimum size=1.5cm, draw=black, dashed] at (A.corner \i) (pents\acc) {};
		\node[circle, minimum size =1cm] at (pents\acc.corner \i) {};
		
		
		
		\pgfmathparse{int(mod(\i,5)+1)}
		\edef\neigh{\pgfmathresult}
		
		
		\pgfmathparse{int(abs(mod(\i-\lados-1,\lados)))}
		\edef\labeli{\pgfmathresult}
		
		\node[] at (pents\acc.center) {\labeli};
		
		\node[circle,shading=ball, ball color=white, draw=black, minimum width=0.3cm,inner sep=0.1cm,opacity=0.2] at (pents\acc.corner \neigh) {};
		}
		
    \node[circle] at (pents1.corner 1) {};
    
    
	
\end{scope}

\end{tikzpicture}


\end{document}
    
